% \iffalse meta-comment
% 
% tikz-eventB.ins
% 
% Copyright (C) 2016 by Thai Son Hoang <T.S.Hoang at ecs dot soton dot ac dot uk>
% --------------------------------------------------------------------
% 
% This file may be distributed and/or modified under the
% conditions of the LaTeX Project Public License, either version 1.3
% of this license or (at your option) any later version.
% The latest version of this license is in:
% 
%      http://www.latex-project.org/lppl.txt
% 
% and version 1.3 or later is part of all distributions of LaTeX 
% version 2003/12/01 or later.
% 
% This work has the LPPL maintenance status "author-maintained".
% 
% This work consists of the files eventB.dtx, eventB.ins,
% the derived file eventB.sty, the generated documentation
% eventB.pdf, and some sample documents.
% 
% \fi
% 
% \iffalse
%<tikz-eventB>%%%%% BEGIN Identification part %%%%%
%<tikz-eventB>% ========================
%<tikz-eventB>\NeedsTeXFormat{LaTeX2e}\relax
%<tikz-eventB>\ProvidesPackage{tikz-eventB}
%<tikz-eventB>    [2014/06/12 v0.0.1 Package for typesetting diagrams of Event-B models] 
%<tikz-eventB>%%%%% END Identification part %%%%%
% 
%<*driver> 
\documentclass[a4paper]{ltxdoc}
\usepackage[colour]{eventB}
\usepackage{tikz-eventB}
\EnableCrossrefs
% ^^A Enable either \CodelineIndex or \PageIndex
\CodelineIndex
% ^^A\PageIndex
\RecordChanges

\begin{document}
\DocInput{tikz-eventB.dtx}
\end{document}
%</driver> 
% \fi
% 
% \CheckSum{0}
% 
% \CharacterTable
% {Upper-case    \A\B\C\D\E\F\G\H\I\J\K\L\M\N\O\P\Q\R\S\T\U\V\W\X\Y\Z
% Lower-case    \a\b\c\d\e\f\g\h\i\j\k\l\m\n\o\p\q\r\s\t\u\v\w\x\y\z
% Digits        \0\1\2\3\4\5\6\7\8\9
% Exclamation   \!     Double quote  \"     Hash (number) \#
% Dollar        \$     Percent       \%     Ampersand     \&
% Acute accent  \'     Left paren    \(     Right paren   \)
% Asterisk      \*     Plus          \+     Comma         \,
% Minus         \-     Point         \.     Solidus       \/
% Colon         \:     Semicolon     \;     Less than     \<
% Equals        \=     Greater than  \>     Question mark \?
% Commercial at \@     Left bracket  \[     Backslash     \\
% Right bracket \]     Circumflex    \^     Underscore    \_
% Grave accent  \`     Left brace    \{     Vertical bar  \|
% Right brace   \}     Tilde         \~}
% 
% 
% \changes{v0.0.1}{2016/06/12}{Initial version}
% 
% \DoNotIndex{\DeclareOption, \ExecuteOptions, \ProcessOptions, \RequirePackage}
% \DoNotIndex{\\,\ }
% \DoNotIndex{\hspace,\quad,\setlength,\newlength}
% \DoNotIndex{\newcommand, \renewcommand, \newenvironment}
% \DoNotIndex{\normalsize, \small, \footnotesize, \scriptsize}
% \DoNotIndex{\textsc}
% \DoNotIndex{\ensuremath, \mathbf, \mathit, \mathsf}
% \DoNotIndex{\begin, \end, \fbox, \fboxsep, \hbox}
% \DoNotIndex{\textcolor, \colorlet}
% \DoNotIndex{\xspace}
% \DoNotIndex{\def, \csname, \endcsname, \expandafter, \ifstrequal}
% \DoNotIndex{\land, \widehat}
% \DoNotIndex{\B@tmp@length}
%
% \GetFileInfo{tikz-eventB.sty}
%
% \title{The \textsf{tikz-eventB} package\thanks{This document corresponds to \textsf{tikz-eventB}~\fileversion, dated~\filedate.}}
% \author{Thai Son Hoang \\ University of Southampton, U.K. \\ \texttt{<T.S.Hoang at ecs dot soton dot ac dot uk>}}
% \date{\today}
% 
% \maketitle
% 
% ^^A %%%%% Abstract %%%%%
% \begin{abstract}
%   This class provides facilities for typesetting diagrams for Event-B models.  It was
%   developed at the University of Southampton, U.K.
% \end{abstract}
% 
% ^^A %%%%% Table of contents %%%%%
% \tableofcontents
% 
% ^^A %%%%% Introduction %%%%%
% \section{Introduction}
% 
% This package was developed in order to ease the typesetting of
% diagrams for Event-B models in \LaTeX{} using TikZ.
% 
% ^^A %%%%% Usage %%%%%%
% \section{Usage}
% See |sample-tikz-eventB.tex| for an example of how to use the package.
%
% \StopEventually{
% \PrintIndex
% \PrintChanges
% }
%
% ^^A %%%%% Implementation %%%%%
% \section{Implementation}
%
% \subsection{Package Loading}
% \label{sec:package-loading}
% We begin by loading the required package |tikz| and |eventB|.
% \iffalse ^^A BEGIN Produce comments only in the resulting class file
%<tikz-eventB>
%<tikz-eventB>%%%%% BEGIN Package loading %%%%%
%<tikz-eventB>% ========================
% \fi^^A END Produce comments only in the resulting class file
%    \begin{macrocode}
\RequirePackage{tikz}
\RequirePackage{eventB}
%    \end{macrocode}
% \iffalse ^^A BEGIN Produce comments only in the resulting class file
%<tikz-eventB>%%%%% END Package loading %%%%%
% \fi^^A END Produce comments only in the resulting class file
%
% \subsection{Commands for Creating Diagrams of Event-B Models}
% \label{sec:comm-pretty-print}

% \iffalse ^^A BEGIN Produce comments only in the resulting class file
%<tikz-eventB>
%<tikz-eventB>%%%%% BEGIN Commands for Creating Diagrams Event-B Models %%%%%
%<tikz-eventB>% ========================
% \fi^^A END Produce comments only in the resulting class file
%
% \begin{macro}{Bdiagram}
%   \changes{v0.0.1}{2016/06/12}{Added}
%   The |Bdiagram| environment for creating diagrams for \eventB models which is the 
%   same as the |tikzpicture| environment.  The environment has an optional argument 
%   which will be passed to the |tikzpicture| environment.
%   \iffalse ^^A BEGIN Produce comments only in the resulting class file
%<tikz-eventB> 
%<tikz-eventB>% Bdiagram environment the same as the tikzpicture
%<tikz-eventB>% environment.
%<tikz-eventB>% Arguments:
%<tikz-eventB>% 1. (Optional) These option will be passed to the
%<tikz-eventB>% tikzpicture environment
% \fi^^A END Produce comments only in the resulting class file
%    \begin{macrocode}
\newenvironment{Bdiagram}[1][]
{\begin{tikzpicture}[#1]}
{\end{tikzpicture}}
%    \end{macrocode}
% \end{macro}^^A END \Bdiagram

% \paragraph{Drawing Event-B Components}
% \begin{macro}{\tikzMch}
%   The \eventB components, i.e., machines and contexts are drawed using |\tikzMch| 
%   and |\tikzCtx| commands.
%   \changes{v0.0.1}{2016/06/12}{Added}
%   \iffalse ^^A BEGIN Produce comments only in the resulting class file
%<tikz-eventB> 
%<tikz-eventB>% Macro for drawing a node for a machine.
%<tikz-eventB>%
%<tikz-eventB>% Arguments:
%<tikz-eventB>% 1. (Optional) The label of the node.
%<tikz-eventB>% 2. The x-coordinate.
%<tikz-eventB>% 3. The y-coordinate.
%<tikz-eventB>% 4. The name of the machine.
%<tikz-eventB>%
%<tikz-eventB>% USAGE:
%<tikz-eventB>% \tikzMch[mch]{0}{1}{M} will produce a node label ``mch'' at position 
%<tikz-eventB>% (0,1). The content of the node, i.e., M will be put inside a \Bmch{}
%<tikz-eventB>% automatically.
% \fi^^A END Produce comments only in the resulting class file
%    \begin{macrocode}
\newcommand{\tikzMch}[4][]{
  \draw(#2,#3)
  node[draw, inner sep = 2ex, minimum width = 4em](#1){\Bmch{#4}};
}
%    \end{macrocode}
% \end{macro}^^A END \tikzMch
% \begin{macro}{\tikzCtx}
%   \changes{v0.0.1}{2016/06/12}{Added}
%   \iffalse ^^A BEGIN Produce comments only in the resulting class file
%<tikz-eventB>
%<tikz-eventB>% Macro for drawing a node for a context.
%<tikz-eventB>%
%<tikz-eventB>% Arguments:
%<tikz-eventB>% 1. (Optional) The label of the node.
%<tikz-eventB>% 2. The x-coordinate.
%<tikz-eventB>% 3. The y-coordinate.
%<tikz-eventB>% 4. The name of the context.
%<tikz-eventB>%
%<tikz-eventB>% USAGE:
%<tikz-eventB>% \tikzMch[ctx]{0}{1}{C} will produce a node label ``ctx'' at position 
%<tikz-eventB>% (0,1). The content of the node, i.e., C will be put inside a \Bctx{}
%<tikz-eventB>% automatically.
% \fi^^A END Produce comments only in the resulting class file
%    \begin{macrocode}
\newcommand{\tikzCtx}[4][]{
  \draw(#2,#3)
  node[draw, rounded corners, inner sep = 2ex, minimum width = 4em](#1){\Bctx{#4}};
}
%    \end{macrocode}
% \end{macro}^^A END \tikzCtx
%
% \iffalse ^^A BEGIN Produce comments only in the resulting class file
%<tikz-eventB>
%<tikz-eventB>%%%%% END Commands for Creating Diagrams Event-B Models %%%%%
%<tikz-eventB>% ========================
% \fi^^A END Produce comments only in the resulting class file
%
% \paragraph{Drawing Relationships between Event-B Components}
% The relationship between \eventB components are drawn using the coordinates of the
% components.  Note that the coordinate of the system can be specified relatively to the
% optional label given to components as in commands |\tikzMch| or |\tikzCtx|.
%
% \begin{macro}{\tikzSees}
%   \changes{v0.0.1}{2016/06/12}{Added}
%   The |\tikzSees| command draws the \Bsees relationship between a machine and a 
%   context.
%   \iffalse ^^A BEGIN Produce comments only in the resulting class file
%<tikz-eventB>
%<tikz-eventB>% Macro for drawing a SEES relationship between a machine and a context.
%<tikz-eventB>%
%<tikz-eventB>% Arguments:
%<tikz-eventB>% 1. (Optional) The style for drawing, passed as it is
%<tikz-eventB>%     to the underlying \draw command
%<tikz-eventB>% 2. The coordinate of a machine.
%<tikz-eventB>% 3. The coordinate of a context.
%<tikz-eventB>%
%<tikz-eventB>% USAGE:
%<tikz-eventB>% \tikzSees{mch}{ctx} will produce an arrow with the label ``\Bsees'' 
%<tikz-eventB>% from the node labelled ``mch'' to the node labelled ``ctx''.
%<tikz-eventB>% The labels must be created as (optional arguments) before using
%<tikz-eventB>% the \tikzMch and \tikzCtx commands.
% \fi^^A END Produce comments only in the resulting class file
%    \begin{macrocode}
\newcommand{\tikzSees}[3][]{
  \draw[->, #1] (#2) --node[fill=white]{\Bsees} (#3);
}
%    \end{macrocode}
% \end{macro}^^A END \tikzSees
%
% \begin{macro}{\tikzRefines}
%   \changes{v0.0.1}{2016/06/12}{Added}
%   The |\tikzRefines| command draws the \Brefines relationship between a (concrete) 
%   machine and an (abstract) machine
%   \iffalse ^^A BEGIN Produce comments only in the resulting class file
%<tikz-eventB>
%<tikz-eventB>% Macro for drawing a REFINES relationship between a (concrete) machine and an (abstract) machine.
%<tikz-eventB>%
%<tikz-eventB>% Arguments:
%<tikz-eventB>% 1. (Optional) The style for drawing, passed as it is
%<tikz-eventB>%     to the underlying \draw command
%<tikz-eventB>% 2. The coordinate of the concrete machine.
%<tikz-eventB>% 3. The coordinate of the abstract machine.
%<tikz-eventB>%
%<tikz-eventB>% USAGE:
%<tikz-eventB>% \tikzRefines{cnc}{abs} will produce an arrow with the label ``\Brefines'' 
%<tikz-eventB>% from the node labelled ``cnc'' to the node labelled ``abs''.
%<tikz-eventB>% The labels must be created as (optional arguments) before using
%<tikz-eventB>% rhe \tikzMch command.
% \fi^^A END Produce comments only in the resulting class file
%    \begin{macrocode}
\newcommand{\tikzRefines}[3][]{
  \draw[->, #1] (#2) --node[fill=white]{\Brefines} (#3);
}
%    \end{macrocode}
% \end{macro}^^A END \tikzRefines
%
% \begin{macro}{\tikzRefinesTransitive}
%   \changes{v0.0.1}{2016/06/12}{Added}
%   The |\tikzRefinesTransitive| command draws the transtive \Brefines relationship between a (concrete) 
%   machine and an (abstract) machine
%   \iffalse ^^A BEGIN Produce comments only in the resulting class file
%<tikz-eventB>
%<tikz-eventB>% Macro for drawing a transtive REFINES relationship between a (concrete) machine and an (abstract) machine.
%<tikz-eventB>%
%<tikz-eventB>% Arguments:
%<tikz-eventB>% 1. (Optional) The style for drawing, passed as it is
%<tikz-eventB>%     to the underlying \draw command
%<tikz-eventB>% 2. The coordinate of the concrete machine.
%<tikz-eventB>% 3. The coordinate of the abstract machine.
%<tikz-eventB>%
%<tikz-eventB>% USAGE:
%<tikz-eventB>% \tikzRefinesTranstive{cnc}{abs} will produce an dashed arrow with the label ``\Bextends'' 
%<tikz-eventB>% from the node labelled ``cnc'' to the node labelled ``abs''.
%<tikz-eventB>% The labels must be created as (optional arguments) before using
%<tikz-eventB>% the \tikzMch command.
% \fi^^A END Produce comments only in the resulting class file
%    \begin{macrocode}
\newcommand{\tikzRefinesTransitive}[3][]{
  \draw[->, dashed,#1] (#2) --node[fill=white]{\Brefines} (#3);
}
%    \end{macrocode}
% \end{macro}^^A END \tikzRefinesTranstive
%
% \begin{macro}{\tikzExtends}
%   \changes{v0.0.1}{2016/06/12}{Added}
%   The |\tikzExtends| command draws the \Bextends relationship between a (concrete) 
%   context and an (abstract) context.
%   \iffalse ^^A BEGIN Produce comments only in the resulting class file
%<tikz-eventB>
%<tikz-eventB>% Macro for drawing a EXTENDS relationship between a (concrete) context and an (abstract) context.
%<tikz-eventB>%
%<tikz-eventB>% Arguments:
%<tikz-eventB>% 1. (Optional) The style for drawing, passed as it is
%<tikz-eventB>%     to the underlying \draw command
%<tikz-eventB>% 2. The coordinate of the concrete context.
%<tikz-eventB>% 3. The coordinate of the abstract context.
%<tikz-eventB>%
%<tikz-eventB>% USAGE:
%<tikz-eventB>% \tikzExtends{cnc}{abs} will produce an arrow with the label ``\Brefines'' 
%<tikz-eventB>% from the node labelled ``cnc'' to the node labelled ``abs''.
%<tikz-eventB>% The labels must be created as (optional arguments) before using
%<tikz-eventB>% the \tikzCtx command.
% \fi^^A END Produce comments only in the resulting class file
%    \begin{macrocode}
\newcommand{\tikzExtends}[3][]{
  \draw[->,#1] (#2) --node[fill=white]{\Bextends} (#3);
}
%    \end{macrocode}
% \end{macro}^^A END \tikzExtends
%
% \begin{macro}{\tikzExtendsTranstive}
%   \changes{v0.0.1}{2016/06/12}{Added}
%   The |\tikzExtends| command draws the transtive \Bextends relationship between a (concrete) 
%   context and an (abstract) context
%   \iffalse ^^A BEGIN Produce comments only in the resulting class file
%<tikz-eventB>
%<tikz-eventB>% Macro for drawing a transtive EXTENDS relationship between a (concrete) conctext and an (abstract) context.
%<tikz-eventB>%
%<tikz-eventB>% Arguments:
%<tikz-eventB>% 1. (Optional) The style for drawing, passed as it is
%<tikz-eventB>%     to the underlying \draw command
%<tikz-eventB>% 2. The coordinate of the concrete context.
%<tikz-eventB>% 3. The coordinate of the abstract context.
%<tikz-eventB>%
%<tikz-eventB>% USAGE:
%<tikz-eventB>% \tikzExtendsTrantive{cnc}{abs} will produce an dashed arrow with the label ``\Bextends'' 
%<tikz-eventB>% from the node labelled ``cnc'' to the node labelled ``abs''.
%<tikz-eventB>% The labels must be created as (optional arguments) before using
%<tikz-eventB>% \tikzCtx command.
% \fi^^A END Produce comments only in the resulting class file
%    \begin{macrocode}
\newcommand{\tikzExtendsTransitive}[3][]{
  \draw[->, dashed,#1] (#2) --node[fill=white]{\Bextends} (#3);
}
%    \end{macrocode}
% \end{macro}^^A END \tikzExtendsTranstive
%
%    \begin{macrocode}
%    \end{macrocode}
% \Finale
\endinput